\maketitle  % command to print the title page with above variables
\setcounter{page}{1}
%---------------------------------------------------------------------
%                  영문 초록을 입력하시오
%---------------------------------------------------------------------
\begin{abstracts}     %this creates the heading for the abstract page
	\addcontentsline{toc}{section}{Abstract}  %%% TOC에 표시
	\noindent{
		Stars are born when matter from interstellar molecular clouds fall to the center to increase the mass of the protostar. Bipolar outflows are formed to remove the excess angular momentum of falling matter. Intensities of outflows are known to be in a close relationship with their bolometric luminosity and evolutionary stages. In this study, data from Institute for Radio Astronomy in the Millimeter Range (IRAM) 30$\,$m Telescope and Taeduk Radio Astronomy Observatory (TRAO) were used. IRAM data were used to map $^{12}\textrm{CO}$ J = 2 - 1 over Orion A molecular cloud. TRAO data were used to map $^{13}\textrm{CO}$ J = 1 - 0 over the same region. Outflows were observed and measured by drawing contour maps and line profiles of  red/blue shifted components. The correlation between a protostar's luminosity and outflow momentum flux have been confirmed. Also, outflows could be detected better if the energy level of the emission line is higher. 
	}
\end{abstracts}

\begin{abstractskor}
	별은 성간분자운의 물질이 중심으로 떨어져 원시성의 질량을 증가시켜야만 탄생된다. 이 과정에서 중심으로 떨어지는 물질의 각운동량을 제거하기 위해 방출류가 발생한다. 여기서 방출류의 세기는 원시성의 진화 단계와 광도와 관련이 있다고 알려져 있다. 이를 새로 관측된 데이터를 사용하여 기존의 연구를 검증해 보려고 한다. 이 연구에서는  Institute for Radio Astronomy in the Millimeter Range (IRAM) 30$\,$m 망원경으로 관측한 $^{12}$CO J = 2 - 1 관측 자료와 대덕 전파 망원경(Taeduk Radio Astronomy Observatory, TRAO)으로 관측한 $^{13}$CO J = 1 - 0 천이 선 자료를 이용하였다. 두 자료 모두 Orion A Cloud 영역을 담고 있다. 빠른 속도를 가진 적색/청색편이된 성분의 contour map을 그려 방출류를 관찰하고 방출류의 세기를 구하였다. 방출류의 세기와 원시성의 광도가 대체적으로 비례한다는 것을 알 수 있었다. 그리고 천이 선의 에너지 준위가 높을수록 방출류를 더 잘 검출할 수 있음을 확인할 수 있었다.
\end{abstractskor}
%----------------------------------------------
%   Table of Contents (자동 작성됨)
%----------------------------------------------
\cleardoublepage
\addcontentsline{toc}{section}{Contents}
\setcounter{secnumdepth}{3} % organisational level that receives a numbers
\setcounter{tocdepth}{3}    % print table of contents for level 3
\baselineskip=2.2em
\tableofcontents


%----------------------------------------------
%     List of Figures/Tables (자동 작성됨)
%----------------------------------------------
\cleardoublepage
\clearpage
\listoffigures	% 그림 목록과 캡션을 출력한다. 만약 논문에 그림이 없다면 이 줄의 맨 앞에 %기호를 넣어서 코멘트 처리한다.

\cleardoublepage
\clearpage
\listoftables  % 표 목록과 캡션을 출력한다. 만약 논문에 표가 없다면 이 줄의 맨 앞에 %기호를 넣어서 코멘트 처리한다.

%%%%%%%%%%%%%%%%%%%%%%%%%%%%%%%%%%%%%%%%%%%%%%%%%%%%%%%%%%%
%%%% Main Document %%%%%%%%%%%%%%%%%%%%%%%%%%%%%%%%%%%%%%%%
%%%%%%%%%%%%%%%%%%%%%%%%%%%%%%%%%%%%%%%%%%%%%%%%%%%%%%%%%%%
\cleardoublepage
\clearpage
\renewcommand{\thepage}{\arabic{page}}
\setcounter{page}{1}



