%-----------------------------------------------------
%  Introduction
%-----------------------------------------------------

\section{Introduction}

Stars are formed in molecular clouds by gravitational accretion. In the early stages of star formation, young stellar objects(YSOs) are still embedded in the molecular clouds, increasing its mass and temperature by accretion of interstellar medium around it. Since the angular momentum is conserved while matter is accreted, matter near the surface of the protostar spins quickly, which stops more accretion. Since angular momentum is removed by jets called bipolar outflows, outflows are observed with size proportional to the mass accreted to the protostar\cite{bontemps1996evolution}. 
It is already known that the accretion rate and the luminosity correlates to each other \cite{kang2013outflow}. The outflow force decreases as protostars evolve from Class 0 to Class I, which means the strength that the protostar pulls interstellar matter decreases as time passes. 
In this study, I will observe the protostars and their outflows of Orion A Cloud. First, I will select the protostars that outflows can be detected from the Spitzer and the Herchel catalogues \cite{megeath2012spitzer, furlan2016herschel}. By using different data sets observed by different observatories and different wavelengths, I will identify the outflows. I will recheck the correlation between the outflow force and its bolometric luminosity. Also, I will compare outflow forces calculated using different wavelengths of light. 