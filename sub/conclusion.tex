\section{결론 및 제언}

Orion A Cloud와 $\rho$ Ophiuchus Cloud 두 영역에서 이전 연구와의 비교를 통해 $^{12}CO$의 높은 천이 선일수록, 좋은 공간 분해능의 관측일수록 방출류 검출이 더 잘 되고 방출류의 세기가 높게 나타나는 것을 통해 방출류가 잘 추적된다는 사실을 알 수 있었다. $^{12}CO$의 더 높은 준위의 분자선으로 관측한 방출류일수록 세기가 더 큰 이유는 다음과 같이 예상된다. 높은 준위의 분자선일수록 분자선의 온도가 더 작다. 방출류는 별 바로 바깥쪽의 외피의 물질을 끌고 나오기 때문에 온도가 비교적으로 높고, 더 높은 에너지의 분자선이 많이 방출된다. 따라서 방출류를 검출하기에는 온도가 높은 분자선인 높은 준위의 분자선일 수록 좋다.\\
또한 본 연구에서는 다양한 별 탄생 영역들의 원시성에 대한 방출류의 세기와 광도의 상관관계에 대해 Orion A Cloud와 ρ Ophiuchus Cloud 두 영역에서도 비슷한 관계를 가지는지 확인하였다. 방출류의 세기와 광도 사이의 상관관계가 나타나는 이유는 다음과 같이 예상된다. 진화가 덜 된 원시성일수록 광도가 크고 수축이 빠르게 일어난다. 수축이 많이 일어나기 때문에 각운동량의 변화가 커서 각운동량을 보존하기 위해 나타나는 방출류가 더 세게 나타나는 것으로 볼 수 있다. 따라서 광도가 클수록 방출류의 세기가 높게 나타난다.\\


이 연구에서는 방출류의 방향 (inclination)이 알려져 있지 않은 경우에는 Takahashi와 마찬가지로 45도로 간주하고 계산을 진행했다. 따라서 각 원시성들에 대한 방출류의 방향을 알게 된다면 더 정확한 계산을 통해 방출류의 세기를 구할 수 있을 것이다. 또한 본 연구에서 각 원시성의 진화단계에 따른 방출류의 세기를 살펴보려 했지만 표본의 수가 적어  광도에 따른 방출류의 세기를 통해 진화단계와의 관계성을 살펴보는데에 어려움이 있었다. Bontemps에서는	$M_{env}$에 따른 방출류의 세기를 관찰해 진화단계에 따른 방출류의 세기와의 관계를 확인하였다. 따라서 본 연구도 추후에 더 많은 원시성들을 관찰하고, 각 원시성에 대한 $M_{env}$값도 구하여 방출류의 세기와 비교하면 더 많은 결과를 도출 할 수 있을 것이다.