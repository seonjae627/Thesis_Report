%% !TeX program = xelatex
%% 부득이하게 pdflatex을 사용해야 할 경우 위의 magic comment를 제거하십시오.

%%%%%%%%%%%%%%%%%%%%%%%%%%%%%%%%%%%%%%%%%%%%%%%%%%%%%%%%%%%%%%%%%%%%%%%%%%%%%%%%%
%%%  LaTeX document class of the thesis of the Gyeonggi Science High School   %%%
%%%  Last edition 2015.11.13 by Chinook Mok                                   %%%
%%%  Continously being modified by gshslatexintro after 2016.02.02.           %%%
%%%  Check the latest version at : latex.gs.hs.kr                             %%%
%%%  Also refer to https://www.facebook.com/gshstexsociety                    %%%
%%%%%%%%%%%%%%%%%%%%%%%%%%%%%%%%%%%%%%%%%%%%%%%%%%%%%%%%%%%%%%%%%%%%%%%%%%%%%%%%%

\documentclass{gshs_thesis}
\graphicspath{{images/}}
% 이곳에 필요한 별도의 패키지들을 적어넣으시오.
%\usepackage{...}
\usepackage{verbatim} % for commment, verbatim environment
\usepackage{spverbatim} % automatic linebreak verbatim environment
%\usepacakge{indentfirst}
\usepackage{tikz}
%\tikzset{
%	image label/.style={
%		every node/.style={
			%fill=black,
			%text=white,
%			font=\sffamily\scriptsize,
%			anchor=south west,
%			xshift=0,
%			yshift=0,
%			at={(0,0)}
%		}
%	}
%}
\usepackage{amsmath}
\usepackage{amsfonts}
\usepackage{amssymb}
\usepackage{float}
\usepackage{graphicx}
\usepackage{tabularx}
\usepackage{multirow}
\usepackage{booktabs}
\usepackage{longtable}
\usepackage{gensymb}
%\usepackage{subcaption}
%\usepackage{floatrow}
%\usepackage{pict2e}

\usepackage{pgfplots}
\pgfplotsset{
	compat=newest,
	label style={font=\sffamily\scriptsize},
	ticklabel style={font=\sffamily\scriptsize},
	legend style={font=\sffamily\tiny},
	major tick length=0.1cm,
	minor tick length=0.05cm,
	every x tick/.style={black},
}

\usetikzlibrary{shapes}
\usetikzlibrary{plotmarks}
\usepackage{listings}
\usepackage{hologo}
\usepackage{makecell}

\lstset{
	basicstyle=\small\ttfamily,
	columns=flexible,
	breaklines=true
}

\citation
\bibdata



% -----------------------------------------------------------------------
%                   이 부분은 수정하지 마시오.
% -----------------------------------------------------------------------
\titleheader{졸업논문청구논문}
\school{과학영재학교 경기과학고등학교}
\approval{위 논문은 과학영재학교 경기과학고등학교 졸업논문으로\\
졸업논문심사위원회에서 심사 통과하였음.}
\chairperson{심사위원장}
\examiner{심사위원}
\apprvsign{(인)}
\korabstract{초 록}
\koracknowledgement{감사의 글}
\korresearches{연 구 활 동}

%: ----------------------------------------------------------------------
%:                  논문 제목과 저자 이름을 입력하시오
% ----------------------------------------------------------------------
\title{Orion A Cloud의 쌍극 방출류의 성질} %한글 제목
\engtitle{Properties of Bipolar Outflows of the Orion A Cloud} %영문 제목
\korname{이 선 재} %저자 이름을 한글로 입력하시오 (글자 사이 띄어쓰기)
\engname{Lee, Seon Jae} %저자 이름을 영어로 입력하시오 (family name, personal name)
\chnname{李 善 在} %저자 이름을 한자로 입력하시오 (글자 사이 띄어쓰기)
\studid{16072} %학번을 입력하시오

%------------------------------------------------------------------------
%                  심사위원과 논문 승인 날짜를 입력하시오
%------------------------------------------------------------------------
\advisor{Park, Kie Hyun}  %지도교사 영문 이름 (family name, personal name)
\judgeone{박 용 선} %심사위원장
\judgetwo{강 동 일}   %심사위원1
\judgethree{박 기 현} %심사위원2(지도교사)
\degreeyear{2019}   %졸업 년도
\degreedate{2018}{7}{21} %논문 승인 날짜 양식

%------------------------------------------------------------------------
%                  논문제출 전 체크리스트를 확인하시오
%------------------------------------------------------------------------
\checklisttitle{[논문제출 전 체크리스트]} %수정하지 마시오
\checklistI{1. 이 논문은 내가 직접 연구하고 작성한 것이다.} %수정하지 마시오
% 이 항목이 사실이라면 다음 줄 앞에 "%"기호 삽입, 다다음 줄 앞의 "%"기호 제거하시오
%\checklistmarkI{$\square$}
\checklistmarkI{$\text{\rlap{$\checkmark$}}\square$}
\checklistII{2. 인용한 모든 자료(책, 논문, 인터넷자료 등)의 인용표시를 바르게 하였다.} %수정하지 마시오
% 이 항목이 사실이라면 다음 줄 앞에 "%"기호 삽입, 다다음 줄 앞의 "%"기호 제거하시오
%\checklistmarkII{$\square$}
\checklistmarkII{$\text{\rlap{$\checkmark$}}\square$}
\checklistIII{3. 인용한 자료의 표현이나 내용을 왜곡하지 않았다.} %수정하지마시오
% 이 항목이 사실이라면 다음 줄 앞에 "%"기호 삽입, 다다음 줄 앞의 "%"기호 제거하시오
%\checklistmarkIII{$\square$}
\checklistmarkIII{$\text{\rlap{$\checkmark$}}\square$}
\checklistIV{4. 정확한 출처제시 없이 다른 사람의 글이나 아이디어를 가져오지 않았다.} %수정하지 마시오
% 이 항목이 사실이라면 다음 줄 앞에 "%"기호 삽입, 다다음 줄 앞의 "%"기호 제거하시오
%\checklistmarkIV{$\square$}
\checklistmarkIV{$\text{\rlap{$\checkmark$}}\square$}
\checklistV{5. 논문 작성 중 도표나 데이터를 조작(위조 혹은 변조)하지 않았다.} %수정하지 마시오
% 이 항목이 사실이라면 다음 줄 앞에 "%"기호 삽입, 다다음 줄 앞의 "%"기호 제거하시오
%\checklistmarkV{$\square$}
\checklistmarkV{$\text{\rlap{$\checkmark$}}\square$}
\checklistVI{6. 다른 친구와 같은 내용의 논문을 제출하지 않았다.} %수정하지 마시오
% 이 항목이 사실이라면 다음 줄 앞에 "%"기호 삽입, 다다음 줄 앞의 "%"기호 제거하시오
%\checklistmarkVI{$\square$}
\checklistmarkVI{$\text{\rlap{$\checkmark$}}\square$} % usepackage 등의 명령어는 여기에.
\usepackage{amsmath}
\usepackage{amssymb}
\usepackage{kotex}
\usepackage{tabu}
\usepackage{booktabs}
\usepackage{siunitx}
\usepackage{graphicx}
\usepackage{subfig}
\usepackage{multirow,makecell}

\usepackage{tocloft}
\setlength{\cftbeforesecskip}{0pt}
\setlength{\cftbeforesubsecskip}{0pt}
\setlength{\cftbeforesubsubsecskip}{0pt}

\begin{document}
%	\renewcommand\baselinestretch{1.2} % line spacing in the paragraph
	\baselineskip=2.2em         % line spacing in the paragraph
	
	\maketitle  % command to print the title page with above variables
\setcounter{page}{1}
%---------------------------------------------------------------------
%                  영문 초록을 입력하시오
%---------------------------------------------------------------------
\begin{abstracts}     %this creates the heading for the abstract page
	\addcontentsline{toc}{section}{Abstract}  %%% TOC에 표시
	\noindent{
		Stars are born when matter from interstellar molecular clouds fall to the center to increase the mass of the protostar. Bipolar outflows are formed to remove the excess angular momentum of falling matter. Intensities of outflows are known to be in a close relationship with their bolometric luminosity and evolutionary stages. In this study, data from Institute for Radio Astronomy in the Millimeter Range (IRAM) 30$\,$m Telescope and Taeduk Radio Astronomy Observatory (TRAO) were used. IRAM data were used to map $^{12}\textrm{CO}$ J = 2 - 1 over Orion A molecular cloud. TRAO data were used to map $^{13}\textrm{CO}$ J = 1 - 0 over the same region. Outflows were observed and measured by drawing contour maps and line profiles of  red/blue shifted components. The correlation between a protostar's luminosity and outflow momentum flux have been confirmed. Also, outflows could be detected better if the energy level of the emission line is higher. 
	}
\end{abstracts}

\begin{abstractskor}
	별은 성간분자운의 물질이 중심으로 떨어져 원시성의 질량을 증가시켜야만 탄생된다. 이 과정에서 중심으로 떨어지는 물질의 각운동량을 제거하기 위해 방출류가 발생한다. 여기서 방출류의 세기는 원시성의 진화 단계와 광도와 관련이 있다고 알려져 있다. 이를 새로 관측된 데이터를 사용하여 기존의 연구를 검증해 보려고 한다. 이 연구에서는  Institute for Radio Astronomy in the Millimeter Range (IRAM) 30$\,$m 망원경으로 관측한 $^{12}$CO J = 2 - 1 관측 자료와 대덕 전파 망원경(Taeduk Radio Astronomy Observatory, TRAO)으로 관측한 $^{13}$CO J = 1 - 0 천이 선 자료를 이용하였다. 두 자료 모두 Orion A Cloud 영역을 담고 있다. 빠른 속도를 가진 적색/청색편이된 성분의 contour map을 그려 방출류를 관찰하고 방출류의 세기를 구하였다. 방출류의 세기와 원시성의 광도가 대체적으로 비례한다는 것을 알 수 있었다. 그리고 천이 선의 에너지 준위가 높을수록 방출류를 더 잘 검출할 수 있음을 확인할 수 있었다.
\end{abstractskor}
%----------------------------------------------
%   Table of Contents (자동 작성됨)
%----------------------------------------------
\cleardoublepage
\addcontentsline{toc}{section}{Contents}
\setcounter{secnumdepth}{3} % organisational level that receives a numbers
\setcounter{tocdepth}{3}    % print table of contents for level 3
\baselineskip=2.2em
\tableofcontents


%----------------------------------------------
%     List of Figures/Tables (자동 작성됨)
%----------------------------------------------
\cleardoublepage
\clearpage
\listoffigures	% 그림 목록과 캡션을 출력한다. 만약 논문에 그림이 없다면 이 줄의 맨 앞에 %기호를 넣어서 코멘트 처리한다.

\cleardoublepage
\clearpage
\listoftables  % 표 목록과 캡션을 출력한다. 만약 논문에 표가 없다면 이 줄의 맨 앞에 %기호를 넣어서 코멘트 처리한다.

%%%%%%%%%%%%%%%%%%%%%%%%%%%%%%%%%%%%%%%%%%%%%%%%%%%%%%%%%%%
%%%% Main Document %%%%%%%%%%%%%%%%%%%%%%%%%%%%%%%%%%%%%%%%
%%%%%%%%%%%%%%%%%%%%%%%%%%%%%%%%%%%%%%%%%%%%%%%%%%%%%%%%%%%
\cleardoublepage
\clearpage
\renewcommand{\thepage}{\arabic{page}}
\setcounter{page}{1}



 % Abstract

	%%%%%%%%%%%%%%%%%%%%%%%%%%%%%%%%%%%%%%%%%%%%%%%%%%%%%%%%%%%
	%%%% Main Document %%%%%%%%%%%%%%%%%%%%%%%%%%%%%%%%%%%%%%%%
	%%%%%%%%%%%%%%%%%%%%%%%%%%%%%%%%%%%%%%%%%%%%%%%%%%%%%%%%%%%

	%-----------------------------------------------------
%  Introduction
%-----------------------------------------------------

\section{Introduction}

Stars are formed in molecular clouds by gravitational accretion. In the early stages of star formation, young stellar objects(YSOs) are still embedded in the molecular clouds, increasing its mass and temperature by accretion of interstellar medium around it. Since the angular momentum is conserved while matter is accreted, matter near the surface of the protostar spins quickly, which prevents more accretion. Since angular momentum is removed by jets called bipolar outflows, outflows are observed with size proportional to the mass accreted to the protostar\cite{bontemps1996evolution}. 
It is already known that the accretion rate and luminosity correlate to each other \cite{kang2013outflow}. The outflow force decreases as protostars evolve from Class 0 to Class I, which means the strength with which the protostar pulls interstellar matter decreases as time passes. 

In this study, I identified the protostars and their outflows of the Orion A Cloud. Aso et al. \cite{aso2000dense} made observations of the $^{12}\textrm{CO (J = 1 - 0)}$ emission and identified 9 CO outflows. Also, Takahashi et al. \cite{takahashi2008millimeter} made observations of the $^{12}\textrm{CO (J = 3 - 2)}$ emmision lines and identified 15 outflows. First, I selected the protostars for which the outflows can be detected from the Spitzer and the Herchel catalogues \cite{megeath2012spitzer, furlan2016herschel}. By using different data sets observed by different observatories and different wavelengths, I identified the outflows. I rechecked the correlation between the outflow force and its bolometric luminosity. Also, I compared the outflow momentum flux calculated using different emission lines. \\
 % Introduction
	
	

\section{분석}

\subsection{identification}

전파로 관측한 데이터는 시선방향에 대해 누적되고, 관찰자에 대한 상대 속도를 가진 물질의 분포를 알려준다. 원시성을 둘러싼 외피는 원시성에 대해 정적이거나 작은 속도로 수축하는 반면에 방출류는 원시성으로부터 두 극지방 방향으로 빠른 속도 성분을 가지고 멀어지므로 방출류의 축이 기울어져 있다면 관찰자를 향해 빠르게 접근하거나 멀어지는 것으로 보인다. 본 연구에서는 물질의 양이 적어 광학적 깊이가 깊기 때문에 외피를 추적하는 $^{13}CO$, $C^{18}O$ 데이터를 이용해서 line을 살펴보았다. 그리고 이를 Gaussian fitting 하여 원시성의 속도 center velocity($v_{cen}$)을 구하고 외피의 속도 분산을 나타내는 반치전폭(FWHM: Full Width Half Maximum)을 구했다. 외피가 아닌 red lobe의 $v_{out}$과 blue lobe의 $v_{in}$ 즉 적분 구간의 시작점은 $v_{cen}$로부터 FWHM만큼 떨어진 지점으로 지정하고, red lobe의 $v_{in}$과 blue lobe의 $v_{out}$은 각 lobe에 대한 오차($\sigma$)이하로 intensity가 떨어지는 지점을 직접 line으로 보면서 정하여 외피가 아닌 적색/청색편이된 성분의 속도 범위를 구하였다. 본 연구에서 관찰한 원시성들에 대한 속도 범위는 부록에 추가하였다.\\
방출류는 분자의 속도가 외피의 속도와 차이가 생겨 도플러 효과에 의해 시선방향으로 외피와는 다른 
주파수를 나타내기 때문에 $^{12}CO$ 방출선은 물질의 양이 많아 광학적 깊이가 얕아 방출류의 정보를 가진다. 또한 외피와의 속도 차이가 큰 영역에서 optically thin하기 때문에 앞에서 구한 식들의 가정에 맞다. 따라서 $^{12}CO$ 데이터의 값을 앞에서 정한 구간에서 적분한 값을 사용하여 적색/청색편이된 성분을 관찰할 수 있는 contour map을 그렸다. 여기서 원시성을 중심으로 서로 반대 방향으로 방출류가 존재하는지 확인하였다. 여기서 보이는 red, blue lobe들이 관찰하고자 하는 원시성으로부터 나오는 방출류인지 확인하기 위해 ccontour map에서 나타나는 red, blue peak와 관찰하고자 하는 원시성의 위치 center에 대해 line을 그렸다. $^{12}CO$와 $^{13}CO$, $C^{18}O$ 천이 선으로 red peak, blue peak, center의 line profile을 그려 관찰하는 원시성으로부터 나온 방출류가 맞는지 확인하였다. 그렇게 확인된 방출류에 대해 기둥 밀도와 방출류의 세기를 구해 분석하였다.\\
여기서 속도 성분 (v + Δv)사이에 있는 물질의 양은 $(\ref{column density})$와 같고, $T_{MB}$는 관측온도$T_A$를 main beam efficiency로 나눠줬을 때의 온도이다. 그리고 방출류의 세기는 $(\ref{FCO})$으로 힘의 단위 [$M_{\odot} km s^{-1} yr^{-1}$]로 계산된다.\cite{Hatchell2}
\\
SED는 http://vizier.u-strasbg.fr/vizier/sed/ 에서 얻은 데이터를 이용하여 그렸다. 관측하고자 하는 원시성으로부터 5 arcsec 위치에서 관측한 다양한 데이터들을 이용하여 SED를 그리고, 그 중 2$\mu m $에서 20$\mu m$ 사이에 있는 데이터들에 대해 선형 추세선을 그려 $\alpha$값을 구했다. 그리고 $\alpha$값을 이용하여 각 대상의 진화단계를 분류하고 알려져있는 classification과 비교해 보았다. 


\section{결과}

\begin{table}[h!]
	\begin{center}
		\begin{tabular}{c|c|c|c|c}
			\toprule
			& \multicolumn{2}{c|}{\textbf{coordinates}} & $\mathbf{L_{bol}}$ & $\mathbf{T_{bol}}$\\
			\textbf{Name} & \textbf{RA} & \textbf{Dec} & ${[L_{\odot}]}$ & $[K]$\\
			\midrule
			\multicolumn{5}{c}{Orion A Cloud}\\
			\midrule
			\centering
			FIR2 & 05:35:24.3 & -5:08:33.3 & 5.68 & 100.6\\
			FIR3 & 05:35:27.5 & -5:09:32.5 & 360.86 & 71.5\\
			FIR6b & 05:35:23.4 & -5:12:03.2 & 21.93 & 54.1\\
			MMS2 & 05:35:18.3 & -05:00:34.8 & 20.11 & 186.3\\
			MMS5 & 05:35:22.4 & -05:01:14.1 & 15.81 & 42.4\\
			MMS9 & 05:35:26.0 & -05:05:42.4 & 8.91 & 38.1\\
			\midrule
			\multicolumn{5}{c}{$\rho$ Ophiuchus Cloud}\\
			\midrule
			\centering
			Elias 32 & 16:27:28.6 & -24:27:19.8 & 1.5 & 620\\
			IRS 46 & 16:27:29.7 & -24:39:16.0 & 0.19 & 280\\
			VLA 1623 & 16:26:26.4 & -24:24:30.9 & 1.0 & 35\\
			BBRCG 24 & 16:27:09.0 & -24:34:08.0 & 0.8 & 1600\\
		\end{tabular}
	\end{center}
	\caption{관측된 원시성들의 정보. 위는 Orion A Cloud에서 관측된 6개의 원시성, 아래는 $\rho$ Ophiuchus Cloud에서 관측된 4개의 원시성이다.}
\end{table}



\begin{figure}[h!]
	\begin{center}
		\begin{tabular}{ccc}
			\includegraphics[height=6.5cm]{Orion_12CO_intmap.png} & \includegraphics[height=6.5cm]{Oph_12CO_intmap.png}
		\end{tabular}
	\end{center}
	\caption{Orion A Cloud $^{12}CO$ intergrated intensity map(좌) $\rho$ Ophiuchus Cloud $^{12}CO$ intergrated intensity map(우). 좌측의 그림은 Orion A Cloud에서 관측한 북쪽 영역의 원시성들의 위치와 이름을 표시하였다. 우측의 그림은 $\rho$ Ophiuchus Cloud의 L1688 영역에서 관측한 원시성들의 위치와 이름, 방출류의 방향을 표시하였다.}
\end{figure}

본 연구에서는 Orion A Cloud의 북쪽 영역에 대해 Takahashi에서 ASTE를 이용한 $^{12}CO$ J=3-2 관측 데이터를 사용하여 발견한 10개의 원시성에 대해 IRAM $^{12}CO$ J=2-1 관측 데이터를 사용해서 6개의 원시성만 방출류를 검출할 수 있었다.\cite{Takahashi}
그리고 $\rho$ Ophiuchus Cloud의 L1688 영역에 대해서는 Marel에서 JCMT를 이용한 $^{12}CO$ J=3-2 관측 데이터를 사용하여 발견한 13개의 원시성에 대해 $^{12}CO$ J=1-0 관측 데이터를 사용한 본 연구에서는 오직 4개의 원시성만 방출류를 검출 할 수 있었다. \cite{Marel}
\\
이후에 나오는 Countour map에서 Orion A Cloud의 경우 10'가 실제 길이로는 1.25pc, $\rho$ Ophiuchus Cloud에서는 10'가 실제 길이로는 0.40pc이다. 그리고 Line profile에서 검은선은 $^{12}CO$ 천이 선, 초록선은 $^{13}CO$ 천이 선, 갈색선은 $C^{18}O$ 천이 선의 line이며 파랑색과 빨강색 점선은 blue, red lobe의 속도 범위를 나타낸 것이다.


\subsection{Orion A Cloud}

\begin{figure}[h!]
	\begin{center}
		\begin{tabular}{ccc}
			\includegraphics[width=5cm]{Orion_12CO2-1_FIR2_rbcontour_400_modified.png} &   \includegraphics[width=5cm]{Orion_12CO2-1_FIR2_line_profile_400.png} &
			\includegraphics[width=5cm]{FIR2_SED.PNG} \\
		\end{tabular}
		\caption{FIR2의 contour map(왼쪽)과 line profile(중간), SED(오른쪽)이다. Red lobe는 contour level이 오차의 60배부터 140배까지 9단계, blue lobe는 100배부터 160배까지 7단계로 나누어서 그림을 그렸다.}
	\end{center}
\end{figure}

\begin{figure}[h!]
	\begin{center}
		\begin{tabular}{ccc}
			\includegraphics[width=5cm]{Orion_12CO2-1_FIR3_rbcontour_400_modified.png} &   \includegraphics[width=5cm]{Orion_12CO2-1_FIR3_line_profile_400.png} &
			\includegraphics[width=5cm]{FIR3_SED.PNG} \\
		\end{tabular}
		\caption{FIR3의 contour map(왼쪽)과 line profile(중간), SED(오른쪽)이다. Red lobe는 contour level이 오차의 40배부터 160배까지 7단계, blue lobe는 60배부터 180배까지 7단계로 나누어서 그림을 그렸다.}
	\end{center}
\end{figure}

\begin{figure}[h!]
	\begin{center}
		\begin{tabular}{ccc}
			\includegraphics[width=5cm]{Orion_12CO2-1_FIR6b_rbcontour_400_modified.png} &   \includegraphics[width=5cm]{Orion_12CO2-1_FIR6b_line_profile_400.png} &
			\includegraphics[width=5cm]{FIR6b_SED.PNG}\\
		\end{tabular}
		\caption{FIR6b의 contour map(왼쪽)과 line profile(중간), SED(오른쪽)이다. Red lobe는 contour level이 오차의 45배부터 105배까지 7단계, blue lobe는 50배부터 110배까지 7단계로 나누어서 그림을 그렸다.}
	\end{center}
\end{figure}

\begin{figure}[h!]
	\begin{center}
		\begin{tabular}{ccc}
			\includegraphics[width=5cm]{Orion_12CO2-1_MMS2_rbcontour_400_modified.png} &   \includegraphics[width=5cm]{Orion_12CO2-1_MMS2_line_profile_400.png} &
			\includegraphics[width=5cm]{MMS2_SED.PNG}\\
		\end{tabular}
		\caption{MMS2의 contour map(왼쪽)과 line profile(중간), SED(오른쪽)이다. Red lobe는 contour level이 오차의 30배부터 40배까지 2단계, blue lobe는 60배부터 130배까지 8단계로 나누어서 그림을 그렸다.}
	\end{center}
\end{figure}

\begin{figure}[h!]
	\begin{center}
		\begin{tabular}{ccc}
			\includegraphics[width=5cm]{Orion_12CO2-1_MMS5_rbcontour_400_modified.png} &   \includegraphics[width=5cm]{Orion_12CO2-1_MMS5_line_profile_400.png} &
			\includegraphics[width=5cm]{MMS5_SED.PNG}\\
		\end{tabular}
		\caption{MMS5의 contour map(왼쪽)과 line profile(중간), SED(오른쪽)이다. Red lobe는 contour level이 오차의 20배부터 30배까지 2단계, blue lobe는 40배부터 90배까지 6단계로 나누어서 그림을 그렸다.}
	\end{center}
\end{figure}

\begin{figure}[h!]
	\begin{center}
		\begin{tabular}{ccc}
			\includegraphics[width=5cm]{Orion_12CO2-1_MMS9_rbcontour_400_modified.png} &   \includegraphics[width=5cm]{Orion_12CO2-1_MMS9_line_profile_400.png} &
			\includegraphics[width=5cm]{MMS9_SED.PNG}\\
		\end{tabular}
		\caption{MMS9의 contour map(왼쪽)과 line profile(중간), SED(오른쪽)이다. Red lobe는 contour level이 오차의 50배부터 140배까지 10단계, blue lobe는 60배부터 200배까지 15단계로 나누어서 그림을 그렸다.}
	\end{center}
\end{figure}

\clearpage
\newpage   
FIR2의 contour map을 살펴보면 N-S 방향으로 강한 방출류가 보인다. 방출류의 크기가 약 30 arcsec정도로 다른 방출류들보다 훨씬 작다. 본 연구의 결과가 Takahashi보다 약 3배 더 약하게 나타났다. Aso는 이 천체에 대하여 분석을 하지 않았다. SED의 기울기 $\alpha$는 3.12으로 Class I으로 분류하였다. Furlan에서도 Class I으로 분류하였다.\cite{HerschelFurlan} 

FIR3의 contour map을 살펴보면  red lobe와 blue lobe의 중심이 거의 같은 위치에 있다. 방출류가 거의 시선방향과 나란하다는 것을 알 수 있다. 본 연구의 결과가 Takahashi보다 약 20배 더 약하게 나타났다. Takahashi는 red와 blue lobe를 각각 2개씩 관측했다. Aso는 이 천체에 대하여 분석을 하지 않았다.  SED의 기울기 $\alpha$는 1.51으로 Class I으로 분류하였다. Furlan에서도 Class I으로 분류하였다.\cite{HerschelFurlan} 

FIR6b의 contour map을 살펴보면 주변의 다른 별들로 인해서 방출류 구조 말고 다른 별들에서 나온 방출류로 인한 선들이 많이 보인다. NW-SE 방향으로 방출류가 관측이 된다. 본 연구의 결과가 Takahashi보다 약 4배 더 약하게 나타났다. Aso는 이 천체에 대하여 분석을 하지 않았다. SED의 기울기 $\alpha$는 1.20으로 Class I으로 분류하였다. Furlan에서는 Class 0으로 분류하였다.\cite{HerschelFurlan} 

MMS2의 contour map을 살펴보면 red lobe와 blue lobe가 둘 다 별을 기준으로 동쪽에 있는 특이한 모양을 하고 있다. SW 방향에 보이는 방출류 구조는 MMS5로 인한 방출류이다. 아마 이것에 의해 방출류가 영향을 받아 치우쳐졌을 가능성이 있다. 본 연구의 결과가 Takahashi보다 약 4배 더 약하게 나타났다. J=1-0 을 사용한 Aso보다 약 1.8배 더 강하게 나타났다. Aso는 MMS2, MMS3, MMS4 세 개의 원시성을 하나의 원시성으로 간주하고 방출류를 계산했다.\cite{Aso} SED의 기울기 $\alpha$는 1.23으로 Class I으로 분류하였다. Furlan에서는 Flat으로 분류하였다.\cite{HerschelFurlan} 

MMS5의 contour map을 살펴보면 E-W 방향으로 방출류가 관측이 된다. blue lobe가 red lobe보다 더 강하게 관측된다. 본 연구의 결과가 Takahashi보다 약 4배 더 약하게 나타났다. Aso보다 약 10\% 강하게 나타났다.\cite{Aso} SED의 기울기 $\alpha$가 3.17로 Class I로 분류되어야 하지만 $2.2\mu m$와 $20 \mu m$ 사이의 관측 데이터의 값이 $10^{-15}$ Wm$^{-2}$정도로 매우 작게 나타났기 때문에 Class 0으로 분류하였다. Furlan에서는 Class I으로 분류하였다.\cite{HerschelFurlan} 

MMS9의 contour map을 살펴보면  E-W 방향으로 강한 방출류가 나오는 것을 볼 수 있다. Takahashi(2008)에서는 red lobe를 두개 관측했는데, 본 연구의 관측 자료에서도 blue lobe의 중심 부근에서 작은 red lobe가 존재하는것을 알 수 있다. 그 세기는 main red lobe보다 10배 정도 더 작은것으로 관측되었다. 본 연구의 결과가 Takahashi보다 약 20배 더 약하게 나타났다. Takahashi는 2개의 red lobe를 관측했다.  Aso의 결과에 비해서는 약 6.7배 약하게 나타났다.\cite{Aso} SED의 기울기 $\alpha$는 1.53으로 Class I으로 분류하였다. Furlan에서는 Class 0으로 분류하였다.\cite{HerschelFurlan} 
\\

각 원시성들의 line profile에 나타난 $^{13}CO$와 $C^{18}O$ 천이 선을 보면 red peak, center, blue peak 세 지점에서 모두 비슷한 개형이 나타났다. 따라서 각 원시성 주변의 red, blue lobe가 본 연구에서 관찰한 원시성으로부터 나온 방출류라는 것을 알 수 있다. 그리고 SED로부터 분류한 진화단계가 일부 원시성들은 Furlan의 결과와 다르게 나타났는데, Furlan은 각 원시성들을 bolometric temperature을 기준으로 분류했기 때문에 본 연구에서 SED를 이용해서 구한 classification과 차이가 있을 수 있다.

\clearpage
\newpage   


\subsection{방출류의 세기}
\begin{table}[h]
	\begin{center}
		\begin{tabular}{c|c|c|c}
			\toprule
			\textbf{Name} &$\mathbf{F_{R}}$ & $\mathbf{F_{B}}$ & $\mathbf{F_{CO}}$\\
			& \multicolumn{3}{c}{[M$_{\odot}$ km s$^{-1}$ yr$^{-1}$]}\\
			\midrule
			\multicolumn{4}{c}{Orion A Cloud}\\
			\midrule
			FIR2 & 1.14E-05 & 3.28E-05 & 4.42E-05\\
			FIR3 & 4.77E-03 & 7.43E-03 & 1.22E-04\\
			FIR6b & 1.13E-05 & 1.18E-05 & 2.31E-05\\
			MMS2 & 1.14E-05 & 4.50E-05 & 5.64E-05\\
			MMS5 & 5.80E-06 & 1.55E-05 & 2.13E-05\\
			MMS9 & 3.67E-06 & 1.09E-05 & 1.46E-05\\
			\midrule
			\multicolumn{4}{c}{$\rho$ Ophiuchus Cloud}\\
			\midrule
			Elias 32 & 1.77E-06 & 1.01E-05 & 1.19E-05\\
			IRS 46 & 4.56E-07 & 7.14E-07 & 1.17E-06\\
			VLA 1623 & 2.42E-06 & 3.15E-06 & 5.57E-06\\
			BBRCG 24 & 3.78E-07 & 8.19E-07 & 1.20E-06\\
		\end{tabular}
	\end{center}
	\caption{관측된 원시성들의 방출류의 세기}
\end{table}

표에서 $F_R$와 $F_B$는 각각 red, blue lobe의 방출류의 세기를 구한것이다.(\ref{FCO}) $F_{CO}$는 두 값을 더한 값으로 원시성이 방출해내는 총 방출류의 세기이다. 두 영역의 방출류를 비교해 보면 $\rho$ Ophiuchus Cloud은 Orion A Cloud보다 질량이 작고 광도가 낮은 별들이 탄생하는 영역으로 방출류의 세기 또한 작게 나타났다.\\

\section{토의}

\subsection{기존 연구와 비교}

\begin{figure}[h]
	\begin{center}
		\includegraphics[height=7cm]{COFCO.PNG}
	\end{center}
	\caption{Orion A Cloud와 $\rho$ Ophiuchus Cloud에서 관측한 원시성들에 대해 $^{12}CO$ J=3-2와 J=2-1, J=1-0 방출선으로 구해진 방출류의 세기. Orion A Cloud의 경우 J=2-1, $\rho$ Ophiucus Cloud의 경우 J=1-0이 본 연구에서 구한 방출류의 세기이다.}
\end{figure}

Orion A Cloud를 $^{12}CO$ J=1-0, J=3-2 관측 데이터로 구한 방출류의 세기는 각각 Aso, Takahashi의 결과를 참고하였고, $\rho$ Ophiuchus Cloud를 $^{12}CO$ J=3-2 관측 데이터로 구한 방출류의 세기는 Marel, Nakamura의 결과를 참고하였다.\cite{Aso}\cite{Takahashi}\cite{Marel}\cite{Nakamura} 두 영역에서 모두 MMS9을 제외한 원시성에서는 높은 천이 선일수록 방출류의 세기가 크게 나타났다.
\\
또한 Orion A Cloud 북쪽 영역에 대해 Takahashi에서 $^{12}CO$ J=3-2 관측 데이터를 사용하여 알려진 10개의 원시성 중 본 연구에서 사용한 $^{12}CO$ J=2-1 관측 데이터로는 6개의 방출류만 찾을 수 있었다. 그리고 $\rho$ Ophiuchus Cloud L1688에 대해서도 Marel에서 $^{12}CO$ J=3-2 관측 데이터를 사용하여 발견한 13개의 원시성에 비해 본 연구에서는 $^{12}CO$ J=1-0 관측 데이터를 사용한 결과 오직 4개의 원시성만 관측 가능하였다.\cite{Takahashi}\cite{Marel} 그리고 Orion A Cloud의 경우 본 연구에서 사용한 데이터의 beam size는 11 arcsec로 22 arcsec였던 Takahashi의 데이터보다 더 정밀하기 때문에 온도가 높은 더 안쪽 영역을 추적한 것으로 볼 수 있다. 따라서 높은 $CO$ 천이 선일수록 상대적으로 보다 따뜻한 가스를 추적하고, 방출류를 더 잘 추적한다고 알려져 있는데, Orion A Cloud와 $\rho$ Ophiuchus Cloud 영역에서 이를 확인 할 수 있었다. 하지만 $^{12}CO$ J=2-1, J=1-0 천이 선 또한 $^{12}CO$ J=3-2 천이 선과 비슷한 정도로 충분히 방출류를 추적할 수 있음을 알 수 있다.


\subsection{두 영역의 광도에 따른 방출류의 세기}

\begin{figure}[h]
	\centering
	\includegraphics[height=8cm]{luminosity-outflowforce.JPG}
	\caption{광도-방출류의 세기 관계 그래프}
\end{figure}

이 그래프는 Takahashi, Bontemps, Hogerheijde, Zhang 논문 에서 발견된 다양한 별 탄생 영역 안의 원시성들의 방출류의 세기와 본 연구에서 구한 Orion A Cloud와 $\rho$ Ophiuchus Cloud 영역 안의 원시성들의 방출류의 세기를 가지고 광도와의 관계를 나타낸 것이다.\cite{Takahashi} \cite{Bontemps} \cite{Hogerheijde} \cite{Zhang} 원시성의 광도는 Spitzer와 Herschel 망원경으로 관측된 값들을 사용하였다.\cite{OphDunham} \cite{Spitzer} \cite{HerschelFurlan} Orion A Cloud는 중간 정도 질량의 별들이 형성되는 지역이고 보다 큰 광도와 방출류의 세기를 보인다. $\rho$ Ophiuchus Cloud는 낮은 질량의 별들이 형성되는 지역이고, Orion A Cloud 영역보다는 상대적으로 낮은 광도와 방출류의 세기를 보인다. 관찰한 두 지역의 원시성들에 대해 모두 방출류의 세기와 원시성의 광도가 비례한다는 사실을 재확인 할 수 있었다.

	% Next Section (e.g. Experiment, Linear theory, etc...) 
	% 이외에도 추가로 section마다 파일을 sub 폴더 안에 넣고 여기에서 
	% include 해주면 됩니다.
	% 예시 : methodology.tex을 sub 폴더안에 저장, 이 자리에 
	% \section{Obervations and Data Reduction}

\subsection{Observation Region}
The Orion region consists of two giant molecular clouds, the Orion A and B clouds. This study covers the Orion A Cloud. The Orion A Cloud covers about $29 \, \textrm{deg}^2$ of the sky and its distance is about 450$\,$pc \cite{kounkel2017gould}. The total mass is estimated to be about $10^5 \, M_{\odot}$. It contains several hot molecular cores, such as the BN-KL nebula. It is known that the Orion Cloud was formed by the collision and fragmentation of two giant molecular clouds about 60 million years ago. The effects of the collision can be seen in the present. There is a big velocity gradient along the declination axis. The north side of the Orion A Cloud (OMC 2) shows about 12$\, \rm km\,s^{-1}$ but the south end (L1641) it is about 5$\, \rm km\,s^{-1}$ \cite{schulz2012formation}.

\subsection{Observation Data}
%%% 박기현 수정 시작
The $^{12}\textrm{CO (J = 2 - 1, 230.538\,GHz)}$ data was observed with the IRAM $\textrm{30\,m}$ telescope in Granada, Spain, in 2013. The spatial beamwidth was $\textrm 11^{\prime\prime}$, and the spectral resolution was 0.4$\, \rm km\,s^{-1}$. The noise level was 0.2$\, \rm K$. It only covers the north region of the Orion A cloud \cite{berne2014iram}.

The $^{12}\textrm{CO (J = 1 - 0, 115.271\,GHz})$ data was observed with the NRO 45m telescope in Nobeyama, Japan.

The $^{13}\textrm{CO (J = 1 - 0, 110.201\,GHz})$ and the $\textrm{C}^{18}\textrm{O (J = 1 - 0, 109.782\,GHz})$ data were observed with the $\textrm{13.7\,m}$ telescope at Taeduk Radio Astronomy Observatory (TRAO) in 2017. The spatial beamwidth was $\textrm 45^{\prime\prime}$, and the spectral resolution was 0.05$\, \rm km\,s^{-1}$. The noise level was 0.4$\,$K.
$^{13}\textrm{CO}$ and $\textrm{C}^{18}\textrm{O}$ lines are optically thin lines which can trace most of the matter on the line of sight, contrasting to $^{12}\textrm{CO}$ lines which are so optically opaque that it can only trace the outermost part of the molecular core. In this study, I used TRAO data to determine the protostar's velocity and linewidth which are the kinematic properties of the envelope. Then, I traced the outflow jets using $^{12}$CO data.
%%%박기현 수정 끝

%%%선재 원본
%%%The $^{12}$CO(J = 2 - 1, $230.538\,\rm GHz$) data was observed with the IRAM 30$\,$m telescope in Granada, Spain, in 2013. The spatial beamwidth was 11", and the spectral resolution was 0.4$\, \rm km\,s^{-1}$. The noise level was 0.2$\, \rm K$. It only covers the north region of the Orion A cloud \cite{berne2014iram}. \\
%%%The $^{12}$CO (J = 1 - 0, $115.271\, \rm GHz$) data was observed with the NRO 45m telescope in Nobeyama, Japan.  \\
%%%The $^{13}$CO (J = 1 - 0, $110.201\,$GHz) and the $\textrm{C}^{18}\textrm{O}$(J = 1 - 0, 109.782 GHz) was observed at Taeduk Radio Astronomy Observatory (TRAO) 13.7$\,$m telescope in 2017. The spatial beamwidth was 45", and the spectral resolution was 0.05$\, \rm km\,s^{-1}$. The noise level was 0.4$\,$K.\\
%%%$^{13}$CO and $\textrm{C}^{18}\textrm{O}$ lines are optically thin lines which can trace most of the matter on the line of sight, contrasting to $^{12}$CO lines which are so optically oblique that it can only trace the outermost part of the molecular core. In this study, I used TRAO data to determine the protostar's velocity and linewidth which are the kinematic properties of the envelope. Then, I will trace the outflow jets using $^{12}$CO data.

Figure \ref{fig:map1} shows the data from IRAM and TRAO. The intensity of the $^{12}\textrm{CO}$ data is approximately 10 times stronger than the $^{13}\textrm{CO}$ data.

%\begin{figure}[h]
\begin{figure}[t]
	\begin{center}
		\begin{tabular}{cc}
			\includegraphics[width=0.45\textwidth]{RNE_12CO_Orion} & \includegraphics[width=0.45\textwidth]{Orion_13CO_intmap}
		\end{tabular}
	\end{center}
	\caption{Orion A $^{12}$CO (J = 2 - 1) integrated intensity map (left) $^{13}$CO (J = 1 - 0) integrated intensity map (right).}
	\label{fig:map1}  %%%박기현샘 추가함 (이렇게 넣어야 제데로 ref 됨.)
	%%% 박기현 레이블을 달고, 본문에서 언급하면 latex에서 알아서 같은 페이지로 보내줌.
\end{figure}

\subsection{Identification of Outflows}
Data obtained by observing radio waves were summed over the line of sight, which tells us the distribution of matter with a relative radial velocity to the observer. The envelope around the protostar is static or is contracting slowly towards the protostar itself, but the outflow jets have large velocity components from each pole. If the inclination of the outflows are not zero, it would appear like the jets are moving closer to or further from the observer. 
In this study, $^{13}$CO and $\textrm{C}^{18}\textrm{O}$ lines were used to get the velocity distribution of the protostar. By using Gaussian fitting to the velocity distribution, I calculated the protostar’s central velocity($v_{cen}$) and the full width at half maximum (FWHM). 

Because the emission lines of $^{12}$CO are optically thicker than other lines, it is appropriate to trace the outflows with $^{12}$CO lines. The intervals of the red/blue lobes were obtained by using the central velocity and the FWHM calculated previously. I drew contour maps to find out if bipolar outflows existed with the protostar at the center. To check if the red/blue lobes that were found are outflows from the same protostar I checked the $^{12}$CO, $^{13}$CO, $\textrm{C}^{18}\textrm{O}$ lines from the red peak, blue peak, and the center points. For each outflow confirmed, I calculated the momentum force and the column density of each point.

The column density can be calculated with the following expression:
\begin{align}
	N_{H_2} =& \frac{8\pi \nu^3}{c^3} \frac{1}{(2J_l +3)A}  \notag \\
	& \times \frac{Z(T_{ex})}{\exp(-E_l / kT_{ex})[1-\exp(h\nu / kT_{ex})]} \notag \\
	& \times \frac{\int T_B dV}{J(T_{ex})-J(T_{bg})} \label{column_density}
\end{align}

\begin{equation}
J(T) = \frac{h \nu / k}{\exp(h\nu / kT)-1}
\end{equation}

In equation (\ref{column_density}), $\nu$ is the corresponding frequency of the emission line, $c$ is the speed of light, $J_l$ is the rotational quantum number of the lower energy level, $A$ is the Einstein A coefficient, $Z$ is the partition function, $E_l$ is the rotational energy of the lower energy level, $k$ is the Boltzmann's constant, $T_{ex}$ is the excitation temperature of the transitions, $\int T_B dV$ is the integrated intensity measured, and $T_{bg}$ is the background radiation temperature. I assumed a local thermal equilibrium(LTE) excitation at an outflow temperature of 50$\,$K \cite{takahashi2008millimeter}.
The mass within one beam can be calculated as the following:

\begin{equation}
M_B =  \frac{\pi}{4} D^2 \theta_B ^2 X[\textrm{CO}] N_{\textrm{H}_2} m_{\textrm{H}_2} \label{beam_mass}
\end{equation}

$D$ is the distance to the objects, $\theta_B$ is the beam size, and $m_{\textrm{H}_2}$ is the mass of one hydrogen molecule. $X[\textrm{CO}]$ is the abundance ratio of CO to $\textrm{H}_2$. In this paper, $D = 450\,\textrm{pc}$ and $X[\textrm{CO}] = 10^{-4}$ was used \cite{hatchell2007star}.\\

\subsection{Calculating Momentum Flux}

The momentum flux within one beam is calculated with the following:

\begin{equation}
\dot{P} = \frac{dP}{dt} = \sum_{v} {\frac{M_B (v) (v/ \cos i)}{D\theta_B / (v \tan i)}}
\end{equation}

$v$ is the velocity offset from $v_{cen}$, $M_B (v)$ is the mass within one beam, and $i$ is the inclination within one beam \cite{hatchell2007star}.
Then the momentum flux from individual beams were summed in annuli. 

\begin{equation}
F_{\textrm{CO}} = \sum _{annulus} \frac{2\pi \theta_r}{N_{pix}\theta_B}\dot{P}	
\end{equation}

$N_{pix}$ is the number of pixels in an annulus. $\theta_r$ is the distance between each pixel and the outflow center. $\theta_B$ is the beam size \cite{hatchell2007star, van2013outflow}.
 와 같이 작성
	%%%% 주의
	%%%% 파일이 나뉠 때마다 자동으로 페이지넘김(\clearpage)가 됩니다. 
	%%%% 따라서 subsection을 나누는 용도로는 사용하지 마십시오.
	%%%% \include{sub/experiment} 와 같이...


	\section{결론 및 제언}

Orion A Cloud와 $\rho$ Ophiuchus Cloud 두 영역에서 이전 연구와의 비교를 통해 $^{12}CO$의 높은 천이 선일수록, 좋은 공간 분해능의 관측일수록 방출류 검출이 더 잘 되고 방출류의 세기가 높게 나타나는 것을 통해 방출류가 잘 추적된다는 사실을 알 수 있었다. $^{12}CO$의 더 높은 준위의 분자선으로 관측한 방출류일수록 세기가 더 큰 이유는 다음과 같이 예상된다. 높은 준위의 분자선일수록 분자선의 온도가 더 작다. 방출류는 별 바로 바깥쪽의 외피의 물질을 끌고 나오기 때문에 온도가 비교적으로 높고, 더 높은 에너지의 분자선이 많이 방출된다. 따라서 방출류를 검출하기에는 온도가 높은 분자선인 높은 준위의 분자선일 수록 좋다.\\
또한 본 연구에서는 다양한 별 탄생 영역들의 원시성에 대한 방출류의 세기와 광도의 상관관계에 대해 Orion A Cloud와 ρ Ophiuchus Cloud 두 영역에서도 비슷한 관계를 가지는지 확인하였다. 방출류의 세기와 광도 사이의 상관관계가 나타나는 이유는 다음과 같이 예상된다. 진화가 덜 된 원시성일수록 광도가 크고 수축이 빠르게 일어난다. 수축이 많이 일어나기 때문에 각운동량의 변화가 커서 각운동량을 보존하기 위해 나타나는 방출류가 더 세게 나타나는 것으로 볼 수 있다. 따라서 광도가 클수록 방출류의 세기가 높게 나타난다.\\


이 연구에서는 방출류의 방향 (inclination)이 알려져 있지 않은 경우에는 Takahashi와 마찬가지로 45도로 간주하고 계산을 진행했다. 따라서 각 원시성들에 대한 방출류의 방향을 알게 된다면 더 정확한 계산을 통해 방출류의 세기를 구할 수 있을 것이다. 또한 본 연구에서 각 원시성의 진화단계에 따른 방출류의 세기를 살펴보려 했지만 표본의 수가 적어  광도에 따른 방출류의 세기를 통해 진화단계와의 관계성을 살펴보는데에 어려움이 있었다. Bontemps에서는	$M_{env}$에 따른 방출류의 세기를 관찰해 진화단계에 따른 방출류의 세기와의 관계를 확인하였다. 따라서 본 연구도 추후에 더 많은 원시성들을 관찰하고, 각 원시성에 대한 $M_{env}$값도 구하여 방출류의 세기와 비교하면 더 많은 결과를 도출 할 수 있을 것이다. % Conclusion
	
	%\clearpage  %%% Appendix를 새 페이지에서 시작
\appendix
\renewcommand{\thesection}{\Alph{section}} %%% TOC에 appendix numbering 재설정
\renewcommand{\thesubsection}{\arabic{subsection}}
\renewcommand{\thesubsubsection}{\arabic{subsubsection}}
\titleformat{\section}[hang] {\normalfont\fontsize{21}{21}\selectfont\bfseries}{\Alph{section}.}{1em}{} %%% Appendix section title의 재설정
\titleformat{\subsection}[hang] {\normalfont\fontsize{16}{16}\selectfont\bfseries}{\Alph{section}.\arabic{subsection}.}{1em}{}
\titleformat{\subsubsection}[hang] {\normalfont\fontsize{14}{14}\selectfont}{\Alph{section}.\arabic{subsection}.\arabic{subsubsection}.}{1em}{}
\titleformat{\paragraph}[hang] {\normalfont\fontsize{12}{12}\selectfont\it}{}{1em}{}
\renewcommand{\theequation}{\thesection.\arabic{equation}} %%% Appendix equation numbering 의 재설정
\renewcommand{\thefigure}{\thesection-\arabic{figure}} %%% Appendix figure numbering 의 재설정
\renewcommand{\thetable}{\thesection-\arabic{table}} %%% Appendix table numbering 의 재설정
\setcounter{equation}{0} %%% Appendix equation starting number의 초기화
\setcounter{figure}{0} %%% Appendix figure starting number의 초기화
\setcounter{table}{0} %%% Appendix table starting number의 초기화
\section{부록}
\begin{table}[h!]
	\begin{center}
		\begin{tabular}{c|c|c|c|c|c|c|c|c}
			\toprule
			&\multicolumn{4}{c|}{Previous Work} & \multicolumn{4}{c}{Our Work}\\
			&\multicolumn{2}{c|}{Blue Lobe} & \multicolumn{2}{c|}{Red Lobe} & \multicolumn{2}{c|}{Blue Lobe} & \multicolumn{2}{c}{Red Lobe}\\
			\textbf{Name} & $\mathbf{v_{out}}$ & $\mathbf{v_{in}}$ & $\mathbf{v_{out}}$ & $\mathbf{v_{in}}$&$\mathbf{v_{out}}$ & $\mathbf{v_{in}}$ & $\mathbf{v_{out}}$ & $\mathbf{v_{in}}$\\
			& [km/s] & [km/s] & [km/s] & [km/s] & [km/s] & [km/s] & [km/s] & [km/s] \\ 
			\midrule
			\multicolumn{9}{c}{Orion A Cloud}\\
			\midrule
			FIR2 & -4.1 & 8.9 & 13.2 & 20.8 &-4.1 & 9.4 & 12.9 & 20.8\\
			FIR3 & -4.1 & 8.9 & 13.2 & 25.1 & -4.1 & 9.25 & 13.0 & 25.1\\
			FIR6b & 1.3 & 8.9 & 13.2 & 21.9 & 1.3 & 9.3 & 12.4 & 21.9\\
			MMS2 & 3.5 & 8.9 & 13.2 & 16.5 & 3.5 & 8.8 & 12.8 & 16.5\\
			MMS5 & 1.3 & 8.9 & 13.2 & 21.9 & 1.3 & 9.5 & 13.1 & 21.9\\
			MMS9 & -4.1 & 8.9 & 13.2 & 26.2 & -4.1 & 9.6 & 13.0 & 26.2\\
			\midrule
			\multicolumn{9}{c}{$\rho$ Ophiuchus Cloud}\\
			\midrule
			Elias 32 & -6.7 & 0.8 & 6.0 & 10.3 & -6.7 & 1.2 & 5.3 & 10.3\\
			IRS 46 & -3.7 & 0.4 & 6.5 & 14.1 & -1.2 & 1.1 & 5.9 & 8.4\\
			VLA 1623 & -3 & 10 & 6.5 & 13 & -3 & 1.2 & 5.3 & 9\\
			BBRCG 24 & N.A. & N.A. & N.A. & N.A. & -5 & 1.2 & 5.7 & 9\\
		\end{tabular}
	\end{center}
	\caption{관측한 원시성들의 적색/청색편이 속도 구간}
\end{table} % Appendix가 없는 경우 주석처리하십시오
	
	\begin{thebibliography}{99}
		
		\bibitem{Bontemps} Bontemps, S., et al. "Evolution of outflow activity around low mass embedded young stellar objects." Disks and Outflows Around Young Stars. Springer Berlin Heidelberg, 1996. 270-275.
		\bibitem{Kang} Kang, Seonmi, et al. "Outflow properties of DIGIT embedded sources." 한국천문학회보 38.1 (2013): 51-51.
		\bibitem{Spitzer}Megeath, S. T., et al. "The Spitzer Space Telescope survey of the Orion A and B molecular clouds. I. A census of dusty young stellar objects and a study of their mid-infrared variability." The Astronomical Journal 144.6 (2012): 192.
		\bibitem{OphDunham}Dunham, Michael M., et al. "Young Stellar Objects in the Gould Belt." The Astrophysical Journal Supplement Series 220.1 (2015): 11.
		\bibitem{HerschelFurlan} Furlan, E., et al. "The Herschel Orion protostar survey: spectral energy distributions and fits using a grid of protostellar models." The Astrophysical Journal Supplement Series 224.1 (2016): 5.
		\bibitem{Marel} van der Marel, Nienke, et al. "Outflow forces of low-mass embedded objects in Ophiuchus: a quantitative comparison of analysis methods." Astronomy \& Astrophysics 556 (2013): A76.
		\bibitem{Oriondistance}Kounkel, Marina, et al. "THE GOULD’S BELT DISTANCES SURVEY (GOBELINS). II. DISTANCES AND STRUCTURE TOWARD THE ORION MOLECULAR CLOUDS." The Astrophysical Journal 834.2 (2017): 142.
		\bibitem{Schulz} Schulz, Norbert S. The formation and early evolution of stars: from dust to stars and planets. Springer Science \& Business Media, 2012.
		\bibitem{Berne}Berne, Olivier, Nuria Marcelino, and Jose Cernicharo. "IRAM 30 m Large Scale Survey of 12CO (2-1) and 13CO (2-1) Emission in the Orion Molecular Cloud." The Astrophysical Journal 795.1 (2014): 1
		\bibitem{Hatchell2} Hatchell, Jennifer, et al. "Star formation in Perseus-II. SEDs, classification, and lifetimes." Astronomy \& Astrophysics 468.3 (2007): 1009-1024.
		\bibitem{Takahashi}Takahashi, Satoko, et al. "Millimeter-and Submillimeter-Wave Observations of the OMC-2/3 Region. III. An Extensive Survey for Molecular Outflows." The Astrophysical Journal 688.1 (2008): 344.
		\bibitem{Hogerheijde}Hogerheijde, Michiel R., et al. "Envelope structure on 700 AU scales and the molecular outflows of low-mass young stellar objects." The Astrophysical Journal 502.1 (1998): 315.
		\bibitem{Nakamura} Nakamura, Fumitaka, et al. "Evidence For Cloud-Cloud Collision and Parsec-Scale Stellar Feedback Within the L1641-N Region." The Astrophysical Journal 746.1 (2012): 25.
		\bibitem{Ophsample} Nakamura, Fumitaka, et al. "The Molecular Outflows in the ρ Ophiuchi Main Cloud: Implications For Turbulence Generation." The Astrophysical Journal 726.1 (2010): 46.
		\bibitem{Aso}Aso, Yoshiyuki, et al. "Dense cores and molecular outflows in the OMC-2/3 region." The Astrophysical Journal Supplement Series 131.2 (2000): 465.
		\bibitem{Zhang}Zhang, Qizhou, et al. "Search for CO outflows toward a sample of 69 high-mass protostellar candidates. II. Outflow properties." The Astrophysical Journal 625.2 (2005): 864.
		\bibitem{VLAclass}Ward-Thompson, D., et al. "The immediate environment of the Class 0 protostar VLA 1623, on scales of∼ 50–100 au, observed at millimetre and centimetre wavelengths." Monthly Notices of the Royal Astronomical Society 415.3 (2011): 2812-2817.
		\bibitem{EL32IRS46class}Andre, Philippe, and Thierry Montmerle. "From T Tauri stars to protostars: Circumstellar material and young stellar objects in the rho Ophiuchi cloud." The Astrophysical Journal 420 (1994): 837-862.
		\bibitem{BBRCG24class}Casanova, Sophie, et al. "ROSAT X-ray sources embedded in the rho Ophiuchi cloud core." The Astrophysical Journal 439 (1995): 752-770.
	\end{thebibliography}
	
	\bibliography{bibfile} % 참고문헌
	% BibTeX 코드 쉽게 얻어오는 방법 %
	% Google Scholar 에서 검색한 결과에서 `인용'을 클릭한다.
	% BibTeX 코드를 얻고자 한다면, 하단의 `BibTeX' 을 클릭.
	% 코드가 나온다. Ctrl+A, Ctrl+C로 복사, bibfile에 붙여넣기.
	
	%\include{sub/summary} % Summary
	%(영어로 작성한 학생은 이 부분을 주석 처리하십시오.)
	
	%-----------------------------------------------------
%   감사의 글
%-----------------------------------------------------

%-----------------------------------------------------
%   연구활동 
%-----------------------------------------------------
\begin{researches}
\addcontentsline{toc}{section}{연구활동}  %%% TOC에 표시
\begin{itemize}
\item{2017학년도 교내 R\&E 발표대회에서 우수상 수상}

\end{itemize}
\end{researches} % 감사의 글 & 연구활동
\end{document}